
After building the robot and testing all the implemented behaviours for him, it has been seen that the proposed solution is robust enough to execute any given path.
The robot can work under different ambient light levels thanks to a mounted shield that shadows the sensors, making them less sensitive to ambient light.
The robot can run for over 30 minutes without problems derivated from having a low battery level.

To find a solution for the Sokoban problem an A* algorithm has been devised.
Using the diamond pushes as nodes and the number of steps of the robot to achieve a certain state as cost function, this algorithm calculates the cost of moving each diamond from their position to the nearest goal.
The nodes in the graph are validated using their state.
The individual states are checked for deadlock situations and are used to reduce the graph.

This system takes into account the walls but ignores the diamonds on the map. 
Also the fact that one goal could be the destination of multiple diamonds is not taken in consideration.

The solver can work for any map satisfying a constrained map size.

To load the generated path into the robot, a path converter function was implemented.
This function can convert the generated path to be executable for the robot.

By combining the two parts of this project, it is possible to load a certain sokoban map, find a solution for it and solve fisically the maze with the robot.
