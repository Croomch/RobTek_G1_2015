\subsection{System Behaviours}
The robot was decided to be build upon being able to execute simple behaviours.
The execution of the behaviours would then be controlled from the \textit{Brain},
this is illustrated in figure \ref{fig:behaviourSystem}.

% brain/behaviour diagram
\begin{figure}[H]
\center
\begin{tikzpicture}[node distance=1cm]
%  behaviour box
\node[rectangle,draw,minimum width=11cm, minimum height=4cm, dashed, name=behaviour_box]  at (0,0) {};
\node[name=behaviour] at (-3.7,1.5) {Behaviours};

% Brain node 
\node[rectangle,draw,minimum width=3cm, minimum height=1cm, name=brain, above=of behaviour_box] {Brain};

% behaviours
\node[rectangle,draw,minimum width=1.5cm, minimum height=0.8cm, name=fwd] at (-3.6,0.3) {Fwd};
\node[rectangle,draw,minimum width=1.5cm, minimum height=0.8cm, name=left] at (-1.8,-0.5) {Left};
\node[rectangle,draw,minimum width=1.5cm, minimum height=0.8cm, name=right] at (0,-1.) {Right};
\node[rectangle,draw,minimum width=1.5cm, minimum height=0.8cm, name=follow] at (1.8,-0.5) {Follow};
\node[rectangle,draw,minimum width=1.5cm, minimum height=0.8cm, name=back] at (3.6,0.4) {Push and back};

% arrows inside 
\draw[->] (brain) -- (fwd) ;
\draw[->] (brain) -- (left) ;
\draw[->] (brain) -- (right) ;
\draw[->] (brain) -- (follow) ;
\draw[->] (brain) -- (back) ;
\end{tikzpicture}
\caption{Overview of the systems behaviours.}
\label{fig:behaviourSystem}
\end{figure}

Here the central unit, the \textit{Brain} is responsible for finding the solution to the sokoban problem by invoking the five behaviours.
The \textit{Brain} is hence a combination of both the computer processing the map offline and the Lego-robot online when using the offline generated data to solve the puzzle.

The behaviours are defined in table \ref{tab:behaviourExplained} and based on dividing the behaviours into simple tasks which are easy to program.

\begin{table}[H]
\center
\begin{tabular}{c|l}
Behaviour & Description \\ \hline
Fwd & Makes the robot go straight ahead in the next intersection. \\
Left / Right & The robot turns left/right in the next intersection. \\
Push and back & The robot pushes the can to the next intersection and \\ & goes back to the previous one. \\
Follow & Makes the robot follow the line till next intersection.
\end{tabular}
\caption{Behaviour table.}
\label{tab:behaviourExplained}
\end{table}

\todo[inline]{what about a separate push behaviour?}

With these five behaviours the robot should then be able to navigate around the map and when a tomato can is encountered, push it to its destination.