\subsection{Node Validation}
When the graph is constructed a massive amount of nodes in the possible direction towards the goal are added to the graph.
This does however also require care in order to validate the nodes before they are added to the closed list in order to prevent excessive memory and time consumption.

Each node must hence be tested in the search for the solution in order to prevent duplicates using all the system memory and processing time.
In order to reduce the check uptime when validating a specific state, it was chosen to use a hash-table to validate if a node is present in the closed-set.
For this to work a hashing value for each node is generated.
This is done using the coordinates of the diamonds and the robot at each state.
The position of such was hence combined into a string where the sorted diamonds position is added to the string first and then followed up with the position of the robot.
This is then used as the unique key in the hashing table preventing the table from being filled with duplicates.

When solving the Sokoban problem furthermore a check for validity of the state is performed.
This is done by considering if any of the diamonds has gone into a deadlock.
Since there are many different deadlock situations, it was chosen only to implement the most simple deadlock arising when pushing a diamond into dead square deadlocks.
This prevents some of the paths explored being invalid because once gone into a deadlock the goal is not obtainable from the specific path.
Furthermore a deadlock detection is able to reduce the graph and hence improve the time taken to find the solution on maps where many deadlocks can be entered.

Since only dead square deadlock situations are considered, then these can be computed pre running the path finding.
This is then stored as on a separate map for lookup during runtime.
The dead squares removed are the squares that are corners and pieces along a wall from which the diamond not can be pushed away from neither brought to a goal.
The result of the algorithm is seen on figure \ref{fig:deadlocks}, where figure \ref{fig:orig_deadlock} is the actual map and figure \ref{fig:orig_deadlock} is the map where all deadlock situations are set to walls to indicate the dead square.


\begin{table}[H]
\begin{subtable}{.5\linewidth}
\centering
\begin{tabular}{| *{8}{c} |}
\hline
X & X & X & X & X & X & X & X \\
X & X &   & X & X &   &   & X \\
X &   &   &   &   &   &   & X \\
X & G & X &   &   &   &   & X \\
X & G & X &   & X & X &   & X \\
X &   &   &   &   & G &   & X \\
X &   &   & G &   &   & G & X \\
X & X & X & X & X & X & X & X \\
\hline
\end{tabular}
\caption{Original test map.}
\label{fig:orig_testmap_deadlock}
\end{subtable}
%
\begin{subtable}{.5\linewidth}
\centering
\begin{tabular}{| *{8}{c} |}
\hline
X & X & X & X & X & X & X & X \\
X & X & X & X & X & X & X & X \\
X & X &   &   &   &   &   & X \\
X & G & X &   &   &   &   & X \\
X & G & X &   & X & X &   & X \\
X &   &   &   &   & G &   & X \\
X & X &   & G &   &   & G & X \\
X & X & X & X & X & X & X & X \\
\hline
\end{tabular}
\caption{Deadlock test map.}
\label{fig:orig_deadlock}
\end{subtable}
\caption{Result of using the proposed heuristic calculations on a given map.}
\label{fig:deadlocks}
\end{table}

