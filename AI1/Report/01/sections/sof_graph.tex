
\subsection{Graph Structure}
The graph is used to store the possible route to the solution of the sokoban problem.
It consist of a set of nodes representing states in the solution.
The nodes link to each other depending on the possible ways in which one state can be reached from another.

This implementation specifies a new state to be for each diamond push with intermediate costs as the number of steps to take in order to reach it.
Thus the algorithm returns the optimal solution in terms of the theoretical robot moves.
This also reduces the memory consumption since how the robot reaches a state is not stored, but only the cost of doing so.
%Since only points of interest are stored in the graph and not the redundant information of how the individual steps in between, reducing the memory consumption of the program considerably.
The diamond position and the position of the robot is stored as a set of coordinates on the map in each node.

%One of the necessary data elements in each node, apart from the cost, is the shortest path of executing the change of state in terms of moves in the directions north, south, east and west.
Once a solution to the problem is found, the shortest path from the starting configuration to the goal is then computed.
This is implemented using a string with the characters N, S, E and W.
Upper case was used to indicate that the move required a diamond push and lower case for movement without.



