
\subsection{Color Detector}
The color detector is responsible for figuring out which, if any, brick color passes by the sensor.
It does this by using both input from the ADC Communication and by controlling the three different colors of LEDs.

The color detector component is designed as the state machine in figure \ref{fig:colordetector_fsm}.



\begin{figure}[H]
\centering
\begin{tikzpicture}[node distance=3cm]
\node[circle, minimum width=6cm, name=c] {};

% node circles
\node[initial,accepting,state,name=decide,minimum width=1.8cm] at (c.180)   {Decide}; 
\node[state,name=red,minimum width=1.8cm] at (c.90)   {Red}; 
\node[state,name=green,minimum width=1.8cm] at (c.0)   {Green}; 
\node[state,name=blue,minimum width=1.8cm] at (c.-90)   {Blue}; 

% connections
\draw[->] (decide) to[out=80, in=190] node[midway,above left] {} (red) ;
\draw[->] (red) to[out=-10, in=100] node[midway, below left] {Timeout} (green) ;
\draw[->] (green) to[out=-100, in=10] node[midway,above left] {Timeout} (blue) ;
\draw[->] (blue) to[out=170, in=-80] node[midway,below left] {Timeout} (decide) ;
 
\end{tikzpicture}

\caption{Color detectors state machine.}
\label{fig:colordetector_fsm}
\end{figure}

As seen on figure \ref{fig:colordetector_fsm} the state machine starts in \textit{Decide} from there it cycles around from state to state recording the intensity values for the three different colors.

