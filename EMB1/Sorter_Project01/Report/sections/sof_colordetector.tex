
\subsection{Color Detector}
The color detector is responsible for figuring out which, if any, brick color passes by the sensor.
It does this by using both input from the ADC Communication and by controlling the three different colors of LEDs.

The color detector component is designed as the state machine in figure \ref{fig:colordetector_fsm}.


\todo[inline]{make this state diagram, this is the wrong one!}

\begin{figure}[H]
\centering
\begin{tikzpicture}[node distance=3cm]
\node[circle, minimum width=6cm, name=c] {};

\node[initial,accepting,state,name=start,minimum width=2.2cm] at (c.180)   {Wait}; 

\node[state,name=left,minimum width=2.2cm] at (c.50)   {Left Tray}; 
\node[state,name=right,minimum width=2.2cm] at (c.-50)   {Right Tray}; 


%\node[name=rst] at (-2.5,3) {rst};

\draw[->] (start) to[out=40, in=180] node[midway,above left] {Green} (left) ;
\draw[->] (start) to[out=10, in=230] node[midway,above left] {Timeout} (left) ;

\draw[->] (start) to[out=-10, in=-230] node[midway,below left] {Red} (right) ;
\draw[->] (start) to[out=-40, in=180] node[midway,below left] {Timeout} (right) ;  
\end{tikzpicture}

\caption{Color detectors Finite State Machine.}
\label{fig:colordetector_fsm}
\end{figure}



