\subsection{PC Communication}
The communication to the PC was implemented using uTosNet on the FPGA and a C++ program on the PC.
The data made available to the computer is seen in table \ref{tab:pc_fpga_communication}.

The C++ program created allows writing and reading the uTosNet registers, making the user able to set the treshold light levels for the different bricks and the position of the motor. It allows to read the LED intensity levels recieved on the FPGA from the ADC and the number of bricks of each color counted by the FPGA.

\begin{table}[H]
\centering
\begin{tabular}{|c|c|c|c|c|}
\hline
 & \multicolumn{4}{|c|}{FPGA $\rightarrow$ PC} \\ \hline
& W0 & W1 & W2 & W3 \\ \hline
R0 & 1 & 2 & 4 & 8 \\ \hline
R1 & - & - & - & - \\ \hline
R2 & i\_red & i\_green & i\_blue & - \\ \hline
R3 & bricks\_red & bricks\_green & bricks\_blue & - \\ \hline
 & \multicolumn{4}{|c|}{PC $\rightarrow$ FPGA} \\ \hline
& W0 & W1 & W2 & W3 \\ \hline
R0 & LED\_DEBUG & - & - & - \\ \hline
R1 & low\_threshold\_red & up\_threshold\_red & low\_threshold\_green & up\_threshold\_green \\ \hline
R2 & low\_threshold\_blue & up\_threshold\_blue & - & - \\ \hline
R3 & mp\_left & mp\_center & mp\_right & mp\_overwrite \\ \hline

\end{tabular}
\caption[FPGA and PC data exchange.]{FPGA and PC data exchange. 
Where Rx is the x'th register and Wy is the y'th word in the uTosNet protocol.
For reading the registers with i\_ are intensity values from the ADC for the respective color, bricks\_ is the brick count of the color.
The PC $\rightarrow$ FPGA registers with \_threshold\_ is the upper and lower thresholds for the corresponding brick for the three different colors.
And the mp\_ registers control the position of the motor when it is in the right, left or center position and it include an overwrite which forces it to stay in one of the three states.
}
\label{tab:pc_fpga_communication}
\end{table}


Note that due to the extensive use of uTosNet to modify the different settings used for sorting, then the system will first work when the registers are initialized from the computer.
This is so because the uTosNet module initializes the values to zero on power-up.
See appendix \ref{app:usageguide} for information on how the final code works.


