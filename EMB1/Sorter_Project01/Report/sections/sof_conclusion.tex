
\subsection{Conclusion}
The overall design of the software on the FPGA was given after which each component and its functionality was described.

The topmodule was implemented as an FSM switching depending on the color perceived by the color detector.
The \textit{Color Detector} was likewise implemented as a state machine recording the light intensities for the three different colors at regular intervals.
The \textit{Color Detector} is implemented such that it takes care of the oscillation of the operational amplifier when it is going from one color to another.

An SPI module was created to communicate with the ADC, this was designed such that the \textit{Color Detector} component is capable of requesting a new ADC value on demand.

The uTosNet module was also included in the project.
This was setup such that a set of instructions can be given to the FPGA to adjust its thresholds, motor positions etc. 
Furthermore values can be read from the FPGA to see the amount of bricks that it has sorted for each color and the light intensity values it reads from the ADC.