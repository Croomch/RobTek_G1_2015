This section describes the software used to control the brick sorting system.
First the top module is described and from there the individual components are considered.

The main components and their interface are shown in figure \ref{fig:program_design}.
Apart from the shown connections in figure \ref{fig:program_design}, then a set of connections to transfer variables which can be set from the computer and the clock signal.


\begin{figure}[H]
\centering
\begin{tikzpicture}[node distance=1cm]
% FPGA border
\node[rectangle,draw,minimum width=9cm, minimum height=7cm, dashed, name=FPGA]  {};

% used to align the insides of FPGA
\node[rectangle,minimum width=5cm, minimum height=5cm, name=FPGAaligne] {};

% components of FPGA
\node[rectangle,draw,minimum width=3cm, minimum height=1cm, name=utos] at (FPGAaligne.-135) {uTosNet};
\node[rectangle,draw,minimum width=3cm, minimum height=1cm, name=ad] at (FPGAaligne.45) {\begin{varwidth}{4cm}ADC\\ Communication\end{varwidth}};
\node[rectangle,draw,minimum width=3cm, minimum height=1cm, name=mc] at (FPGAaligne.0) {Motor Control};
\node[rectangle,draw,minimum width=3cm, minimum height=1cm, name=fsm] at (FPGAaligne.180) {FSM};
\node[rectangle,draw,minimum width=3cm, minimum height=1cm, name=color] at (FPGAaligne.135) {Color Detector};

% nodes outside FPGA
\node [left=of utos,name=pc] {PC};
\node [right=of ad,name=adc] {SPI};
\node [left=of color,name=rgb] {RGB(2:0)};
\node [right=of mc,name=servo] {Servo};

% arrows inside FPGA
\draw[<->] (utos) -- node[] {} (fsm) ;
\draw[<-] (color) -- node[above] {10 bit} (ad) ;
\draw[->] (fsm) --  node[above] {2 bit} (mc) ;
\draw[->] (color) -- node[left] {2 bit} (fsm) ;
 
% arrows connected to the outside of FPGA
\draw[<->] (utos) to[out=180, in=0] node[] {} (pc) ;
\draw[->] (adc) to[out=180, in=0] node[] {} (ad) ;
\draw[->] (color) to[out=180, in=0] node[] {} (rgb) ;
\draw[->] (mc) to[out=0, in=180] node[] {} (servo) ;
\end{tikzpicture}

\caption{FPGA Program Design.}
\label{fig:program_design}
\end{figure}



The components of figure \ref{fig:program_design} are responsible for the following tasks.

\paragraph*{FSM}
The \textit{FSM} component is the top module of the system.
It connects the remaining components functionality in order to sort the bricks depending on their color and communicate with the computer.

\paragraph*{Color Detector}
The \textit{Color Detector} is responsible for controlling the LEDs and use the data from the ADC to decide the color of the bricks passing through the system.
This way it is able to connect the light intensity values from the photo-diode and relate them to a specific color.
From there the data is used to decide which brick, if any, is passing the sensor.

\paragraph*{ADC Communication}
This component implements the SPI communication to the ADC and feeds this onwards to the \textit{Color Detector}.


\paragraph*{Motor Control}
The \textit{Motor Control} component is controlling the servo motor using an input signal, deciding which of three positions it should be in.
Either the left- or right-tray or the idle position.

\paragraph*{uTosNet}
UTosNet is the component supplied by SDU which implements the communication with the computer.
This component is hence created such that it is capable of transferring data of importance to the FPGA and change specific settings or gather data from such.



