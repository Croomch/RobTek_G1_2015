
\subsection{Top Module}
The top module is the part of the system which combines all the functionality the FPGA should have.
Its core functionality is implemented as the state machine in figure \ref{fig:topmodule_fsm}.
This state machine is sorting the bricks using the signals generated from the sub components.


\begin{figure}[H]
\centering
\begin{tikzpicture}[node distance=3cm]
\node[circle, minimum width=6cm, name=c] {};

\node[initial,accepting,state,name=start,minimum width=2.2cm] at (c.180)   {Wait}; 

\node[state,name=left,minimum width=2.2cm] at (c.50)   {Left Tray}; 
\node[state,name=right,minimum width=2.2cm] at (c.-50)   {Right Tray}; 


%\node[name=rst] at (-2.5,3) {rst};

\draw[->] (start) to[out=40, in=180] node[midway,above left] {Green, Blue} (left) ;
\draw[->] (start) to[out=10, in=230] node[midway,above left] {Timeout} (left) ;

\draw[->] (start) to[out=-10, in=-230] node[midway,below left] {Red} (right) ;
\draw[->] (start) to[out=-40, in=180] node[midway,below left] {Timeout} (right) ;  
\end{tikzpicture}

\caption{Top module Finite State Machine.}
\label{fig:topmodule_fsm}
\end{figure}


The state machine in figure \ref{fig:topmodule_fsm} waits in the idle mode whenever no brick is or has passed.
When a brick is detected, and the state machine is in the \textit{Wait} state, then it moves to the \textit{Right-} or \textit{Left Tray} state depending on the color of the brick.
When the state is changed, a timer is started and this generated a timeout signal once the time has elapsed.
This timeout results in the state machine returning to its initial \textit{Wait} state when the brick has passed the sorter.
While the state machine is in \textit{Left-} or \textit{Right Tray}, it sends a signal to the \textit{Motor Control} component to move to the left or right side in order to sort the bricks.
Otherwise, the motors are told to return to its initial position, which is when the motor is pointing the sorter towards the falling bricks.
The timeout period before switching back to the \textit{Wait} state was set such that the bricks would have enough time to pass through the sorting mechanism on the slide before the sorter would return to its initial state.


