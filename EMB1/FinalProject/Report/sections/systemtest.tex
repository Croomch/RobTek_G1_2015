The final segway was tested to see if it is capable of balancing itself.
These test were performed using an external power supply connected to the segway.

The PID controller was then tested in a range of different values for the different components of the controller in order to keep it upright.
It was here found that the system is in its current state not able to balance itself.

The system was found to respond appropriately when held a different angles, but without it being able to fully recover.
The faults of the system preventing it may originate from a set of different causes.

Firstly the on board connections connecting the different hardware modules on the segway were not be appropriately chosen for this purpose.
This could be solved by using proper connectors, but these were not available to us during the construction of the system.
This caused some irregular, but with time, more often occurring 'blackouts' on one of the two motors, preventing it from turning.
This also coursed the system to spin around itself rather than try to stabilize itself and hence made testing difficult.

The accelerometer would also need to be fully calibrated to the system in order to give the best prediction of an the inclination of the segway.
This was however, due to time constraint not made possible and was clearly affecting the system.
This was especially noticed when the segway was tested while supporting it and tilting it in the two directions possible.
It here clearly indicated a stronger drive to hold itself at an angle, rather than straight up.
This would cause it to continuously drive forward, if it would be able to hold itself upright, rather than oscillating around the same location.

Possibly one should also consider a different type of controller when stabilizing a segway.
For this project it was chosen to implement the PID because of its simplicity, but it might be worthwhile considering other controllers such as a lead controller.

Another cause of the instability may also be the use of the accelerometer data only.
This is known to be a very noisy signal since it is affected by not only gravity but also vibrations of the motor, vibrations of the module itself, if not fixated probably, the terrain on which the robot drives may also affect the readings depending on if it is a smooth or rough surface.
It should hence preferably be used in combination with the gyroscope data to give a better short term estimate of the angular change.