A segway was constructed using two DC motors, controlled by an H-bridge.
The segway also included a power supply circuit with an switching voltage regulator to convert the given voltage to a 5V supply to the FPGA.
A IMU with an accelerometer and gyroscope was added to the system in order to keep it upright.
The IMU was supplied by the 3.3V port on the FPGA.

The system implemented on the FPGA consisted of a top module controlling the communication with the IMU and distributing and receiving data by the other components.
To facilitate the data exchange with the IMU an SPI module was made to apply to the given standard of the IMU.
A filter module was implemented to filter the data of the accelerometer in order to reduce the noise and convert it into an angle represented with 8 bits.
The filtered data was then further processed in a PID component, converting the estimated angle into a percentage duty cycle for the motors and a direction.
The Motor control unit converted the percentage duty cycle and direction into a PWM signal to the respective parts of the H-Bridge.

Test were performed in order to test the segways performance and fine tune the PID values.
It was here found that the current arrangement is not sufficient to keep it stable and it was hence not possible to achieve a fully functional mini-segway in this project.