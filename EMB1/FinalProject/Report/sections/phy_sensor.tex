
\subsection{Sensors}
In order to control and stabilize the Segway, then a set of sensors are required.
For this project it was decided to both use a set of sensors to control the systems stability and also to implement a set of sensors into the system enabling the tracking of the Segways position.


The tracking of the Segways position is done using an encoder kit\footnote{ \href{https://www.sparkfun.com/products/13339}{Wheel encoder}} mounted on the motors used in this project.
The encoders chosen for this project is an incremental rotary encoder capable to track the movement of the wheels, but not the direction of rotation.
It is hence not directly possible to tell which direction the Segway is moving without combining it with data from other sensors.
The sensors were however deemed good enough for the scope of this project.


The most important sensor of the system is the one used to control the stability of the system in order to hold it upright.
For this an accelerometer or/and a gyroscope can be used.
The advantage of using a gyroscope is that it is reliable in the measure of angular displacement over short periods of time.
When integrating over this it does, however, get troublesome due to drifts when performing the numerical integration.
The accelerometer on the other hand is exposed to a lot of noise from all the surrounding forces working on the system and it is hence only becomes a reliable measure of the systems angle when used with a low-pass filter.
It was therefore decided to use both in the project and combining them using a complimentary filter.
This provides the system with both the short term accuracy of the gyroscope and the long term reliability of the accelerometer.
For this project it was decided to by an integrated circuit
\footnote{ \href{https://www.sparkfun.com/products/13339}{IMU component}}
 including both gyroscope and accelerometer in the same component.
This ensures that the axis of the accelerometer and gyroscope are aligned within the component.



