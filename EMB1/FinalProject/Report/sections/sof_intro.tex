In order to control and stabilize the segway a the communication with the accelerometer/gyroscope must be established.
This must data received must then be processed in order to control the motors according to the segways position.

To do this the system was build up from the modules seen in figure \ref{fig:fpga_sof_design}.




\begin{figure}[H]
\centering

\begin{tikzpicture}[node distance=1cm]
% FPGA border
\node[rectangle,draw,minimum width=9cm, minimum height=7cm, dashed, name=FPGA]  {};

% used to align the insides of FPGA
\node[rectangle,minimum width=5cm, minimum height=5cm, name=FPGAaligne] {};

% components of FPGA
\node[rectangle,draw,minimum width=3cm, minimum height=1cm, name=pid] at (FPGAaligne.-135) {PID};
\node[rectangle,draw,minimum width=3cm, minimum height=1cm, name=communiction] at (FPGAaligne.45) {SPI Module};
\node[rectangle,draw,minimum width=3cm, minimum height=1cm, name=mc] at (FPGAaligne.-45) {Motor Control};
\node[rectangle,draw,minimum width=3cm, minimum height=1cm, name=filter] at (FPGAaligne.180) {Filter};
\node[rectangle,draw,minimum width=3cm, minimum height=1cm, name=segway] at (FPGAaligne.135) {Segway Control};

% nodes outside FPGA
%\node [left=of utos,name=pc] {PC};
\node [right=of communiction,name=spi] {SPI};
%\node [left=of color,name=rgb] {RGB(2:0)};
\node [right=of mc,name=motor] {DC Motors};

% arrows inside FPGA
\draw[<->] (segway) -- node[] {} (communiction) ;
\draw[->] (segway) -- node[left] {16 bit} (filter) ;
\draw[->] (filter) --  node[left] {16 bit} (pid) ;
\draw[->] (segway) -- node[above right] {8 bit} (mc) ;
\draw[->] (pid) --(-0.2,-2.5)--(-0.2,2)-- node[below left] {9 bit} (segway) ;
 
% arrows connected to the outside of FPGA
\draw[<->] (spi) -- node[] {} (communiction) ;
\draw[<->] (mc) -- node[] {} (motor) ;
%\draw[->] (adc) to[out=180, in=0] node[] {} (ad) ;
%\draw[->] (color) to[out=180, in=0] node[] {} (rgb) ;
%\draw[->] (mc) to[out=0, in=180] node[] {} (servo) ;
\end{tikzpicture}

\caption{Block design of the FPGA.}
\label{fig:fpga_sof_design}
\end{figure}

\todo[inline]{very short something about the different blocks after which they are explained in detail in the main part}
